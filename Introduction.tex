
\chapter{Introduction} 
\label{chap:Introduction}

\section{Motivation}
\label{sec:Intro_Motivation}
NAND flash-based solid-state drives (SSDs) are widely used in personal computing
systems as well as mobile embedded systems. However, in enterprise environments,
SSDs are employed in only limited applications because SSDs are not yet cost 
competitive with HDDs. Fortunately, the prices for SSDs have fallen to the 
comparable level of HDDs by continuous semiconductor process scaling 
(e.g., 10 nm-node process) combined with multi-leveling technologies (e.g., 
MLC and TLC). However, the limited endurance of NAND flash memory, which have
declined further as a side effect of the recent advanced device technologies,
is emerging as another major barrier to the wide adoption of SSDs. (NAND endurance
is the ability of a memory cell to endure program/erase (P/E) cycling, and is
quantified as the maximum number $MAX_{P/E}$ of P/E cycles that the cell can tolerate
while maintaining its reliability requirements.) 
For example, although the NAND capacity per die doubles every two years, the 
actual lifetime of SSDs does not increase as much as projected in the past seven
years because the $MAX_{P/E}$ has declined by 70\% during that period~\cite{MooresLaw}.
Since the reduction in the $MAX_{P/E}$ seriously limits the overall lifetime of flash-based SSDs,
the issues concerning the lifetime of SSDs should be properly resolved for SSDs to be commonly 
used in enterprise environments.

Since the Lifetime $L_C$ of an SSD with the total capacity $C$ is proportional to 
the maximum number $MAX_{P/E}$ of P/E cycles, and is inversely proportional to the
total written data $W_{day}$ per day, $L_C$ (in days) can be expressed as follows
~\cite{DPES}(assuming a perfect wear leveling):
\[L_C = \frac{MAX_{P/E} \times C}{W_{day} \times WAF}\],
where $WAF$ is a write amplification factor which represents the efficiency of an FTL
algorithm. Since $MAX_{P/E}$ and $C$ is determined when the device is manufactured,
we should reduce the $W_{day}$ and $WAF$ to improve the lifetime of SSDs.
Many existing lifetime-enhancing techniques have been focused on reducing $WAF$
by increasing the efficiency of an FTL algorithm. For example, by avoiding
unnecessary data copies during garbage collection 
using the multi-stream feature, $WAF$ can be reduced.
In order to reduce $W_{day}$, various system-level techniques were proposed.
For example, data deduplication, data compression, and write traffic 
throttling are such techniques.

Most existing studies, however, are based on the the single I/O layer such as 
block I/O, device driver, and SSD firmware so their effectiveness is limited.
In order for NAND flash-based storage devices to be broadly adopted in
various computing environments, therefore,
new approaches that properly address the lifetime problem of recent
high-density NAND flash memory are highly required.

\begin{comment}
%----------------------------------------------------
As the price-per-byte of NAND flash memory is rapidly decreasing,
NAND flash-based solid-state drives (SSDs) are emerging as a viable high-performance storage solution
for laptops, desktop PCs and high-performance enterprise systems.
%The poor write endurance of NAND flash memory, however, is emerging as one of the main obstacles 
%for wider adoption of SSDs in various computing environments.
%In NAND flash memory, reads and writes are performed in a unit of a page.
%Because of its `erase-before-write' nature,
%a block consisting of multiple pages must be erased before programming (or writing) new data to it.
%Unfortunately, as the semiconductor process is scaled down and the multi-level cell (MLC) technology is introduced,
However, as NAND flash memory technology scales down to 20-nm and below, storing data reliably in NAND flash memory
gets a key design challenge of NAND-based storage systems. 
For example, the number of program/erase (P/E) cycles allowed for each block is significantly reduced in recent triple-level cell (TLC)
NAND technology.
While older 5x-nm single-level cell (SLC) NAND flash memory can support about 10 K P/E cycles, recent 2x-nm TLC NAND flash memory
can barely support about 1 K P/E cycles~\cite{tlc}.
%the number of P/E cycles is reduced to 3K~\cite{mlc1}. %~\cite{mlc1,mlc2}.
%As a result, the reduction in the number of P/E cycles seriously limits the overall lifetime of flash-based SSDs.
%----------------------------------------------------
\end{comment}

\section{Dissertation Goals}
In this dissertation, we propose system-level approaches that improve 
the lifetime of NAND flash-based storage devices, which overcomes the limitations
of the existing techniques. More specifically, our primary goal is to understand 
high-level information of applications or file systems, such as the 
I/O context of dominant I/O activities and data contents of the file system,
and then develop the lifetime improvement approaches that efficiently exploit such
high-level information at various system levels ranging from a system call layer 
to a flash controller.

First, we propose a system-level approach to reduce WAF that exploits
the I/O context of an application to increase the data lifetime prediction
for the multi-streamed SSDs. 
By extracting program contexts during runtime, data-to-stream mapping is done automatically.
Thus, it can effectively separate data with
short lifetimes from data with long lifetimes to improve the efficiency of garbage collection.
Moreover, when data mapped to the same
stream show large differences in their lifetimes,
long-lived data of the current stream are moved to 
its substream during garbage collection.

Second, present a write traffic reduction approach which reduces the amount of
write traffic sent to a storage by eliminating redundant data, therby improving
the lifetime of storage devices. In particular, we increase the overall
deduplication ratio\footnote{The percentage of identified duplicate writes}
by introducing sub-page chunk based on the understanding of file system behavior.
It resolves technical difficulties caused by its finer granularity, i.e., increased memory requirement and read
response time. 

\section{Contributions}
In this dissertation, we present two system-level techniques to improve the 
lifetime of NAND flash-based storage devices. 
The contributions of this dissertation can be summarized as follows:

\begin{itemize}
\item 
We present a fully automatic stream management technique, called \textsf{\small PCStream}, 
for multi-streamed SSDs based on program contexts (PCs).
Since the key motivation behind \textsf{\small PCStream} was 
that data lifetimes should be estimated at a higher abstraction level than LBAs, 
\textsf{\small PCStream} employed a write program context
as a stream management unit.
A program context~\cite{PC, PC2}, which represents a particular execution phase of a program, 
is known to be an effective hint in separating data with different lifetimes~\cite{PCHa}.  
\textsf{\small PCStream} automatically maps an identified program context to a stream.  
Since program contexts can be computed during runtime, 
\textsf{\small PCStream} does not need any manual work.   

\item
We propose a fine-grained deduplication technique for flash-based SSDs, called \textit{FineDedup}.
The proposed FineDedup technique is different from other existing deduplication techniques
in that it increases the likelihood of finding duplicates
by using a finer deduplication unit
which is smaller than a single page (e.g., one fourth of a single page).
With a smaller deduplication unit,
many data segments within a page can be detected as a duplicate one, 
so the amount of data written to flash memory can be reduced regardless of a physical page size.

\item
We implement the proposed techniques in the Linux kernel and our in-house flash storage
prototype. Then, we evaluate their effects on lifetime using various real-world applications.
 
\end{itemize}

\section{Dissertation Structure}
This dissertation is composed of five chapters. The first chapter is the introduction
of theh dissertation, while the last chapter serves as conclusions with a summary
and future work. The three intermediate chapters are organized as follows:

Chapter 2 provides the overall architecture of NAND flash-based storage devices.
We also describe the existing lifetime improvement techniques for flash-based devices,
focusing on multi-streamed SSDs and deduplication techniques which are highly related
to our proposed techniques.

In Chapter 3, we present a new data separation technique, called PCStream, 
for multi-streamed SSDs. We explain the relationship of dominant I/O activities
with data lifetime patterns. 

Chapter 4 introduces a fine-grained deduplication technique, called FineDedup, for NAND
flash-based storage devices. We describe the patterns of duplicate data with in a page.
Finally, we show how effective the proposed technique is in terms of write traffic reduction.

